\chapter{Conclusioni e Sviluppi Futuri}

La realizzazione di un'architettura che risolva il blockchain scalability trilemma è certamente un obiettivo molto ambizioso, in grado di estendere ad un uso globale tecnologie basate sulla blockchain in vari ambiti, come quello delle cripto-valute. \'E stata presentata all'interno di questo lavoro di tesi una definizione teorica di un'architettura in grado di scalare, basata su una parallelizzazione di Algorand, in cui più comitati in un round lavorano parallelamente per la creazione del prossimo blocco ed evitando la comunicazione in broadcast, che impatta fortemente la scalabilità. Si è dimostrato formalmente il suo corretto funzionamento e la scalabilità, con il teorema di correttezza e di scalabilità, rispettivamente. La scalabilità, a differenza di altre soluzione presenti in letteratura, come lo sharding, non impatta, come dimostrato, su decentralizzazione e sicurezza. La dimensione dei comitati, infatti, rimane la medesima anche a fronte di un maggior numero di transazioni. In altre parole il carico è proporzionale al numero dei nodi, mantenendo limitato il tempo di latenza per la conferma delle transazioni candidate. Quindi si può concludere che questo lavoro di tesi è il primo risultato che risolve il blockchain scalability trilemma, considerando la scalabilità di tutti gli elementi architetturali della blockchain e non solo del protocollo di consenso.

Riguardo gli sviluppi futuri, da un punto di vista teorico, è necessaria una prova formale della sicurezza sulla base di un threat model, è importante comprendere l'impatto che ha la rimozione del vincolo di sincronizzazione di tutti i nodi della rete ed il comportamento del sistema in caso di fallimento dei comitati. Inoltre, poiché è stata assunta l'esistenza di un multicast scalabile, è interessante capire come realizzare questo tipo di protocollo, fondamentale per la comunicazione tra i comitati. Infine, si potrebbe estendere il dominio di applicabilità dell'architettura, adesso focalizzato sull'ambito cripto-valute, ad un contesto più generale, introducendo gli smart contract. Da un punto di vista pratico, si può effettuare una simulazione per determinare i parametri reali del sistema.

