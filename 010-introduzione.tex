\chapter{Introduzione}

Le blockchain permettono di creare applicazioni che offrono integrità dei dati, trasparenza e immutabilità, senza alcuna autorità centrale che regoli le transazioni. La decentralizzazione ha aperto nuove strade in molti campi, ad esempio quello finanziario, ma pone delle nuove sfide per assicurare la consistenza su tutti i nodi della rete. Una blockchain pubblica per essere adottata a livello globale deve essere decentralizzata, sicura e scalabile. I tre requisiti, certamente contrastanti, determinano una difficoltà in fase di progettazione di una blockchain che li contempli tutti e tre. Vitalik Buterin, fondatore di Ethereum, ha espresso questa difficoltà nel celebre \textit{blockchain scalability trilemma}. Secondo il trilemma una blockchain può massimizzare solo due delle tre proprietà a scapito della terza.

Negli attuali sistemi di blockchain, ogni nodo riceve e processa tutte le transazioni generate nella rete. Questo è quindi un problema per la scalabilità, poiché ogni nodo rappresenta un collo di bottiglia per la sua banda e potenza computazionale limitata. In letteratura sono stati proposti sia approcci focalizzati sulla scalabilità del consenso, come Algorand~\cite{gilad2017algorand}, sia da un punto di vista architetturale, mediante lo \textit{sharding} (adottato anche nei sistemi di database distribuiti). Lo sharding consiste nel partizionare l'intera rete in tante sottoreti di piccole dimensioni, denominate \textit{shard}, in cui ognuna è responsabile di un sottoinsieme dello stato memorizzato in blockchain. Benché sia una soluzione alla scalabilità, lo sharding penalizza la sicurezza, poiché gli shard sono composti da pochi nodi rispetto a quelli presenti sulla tutta la rete. Inoltre, rende difficile la gestione di transazioni \textit{cross-shard}.

In questo lavoro di tesi, si introduce da un punto di vista teorico un'architettura in grado di scalare proporzionalmente il numero dei nodi con il throughput delle transazioni. La soluzione è architetturale e tiene conto di tutti i suoi aspetti (consenso, comunicazione e storage). In questa nuova proposta, nessun nodo invia in broadcast le transazioni che genera e non è obbligato a ricevere e processare tutte le transazioni create in un round. La scalabilità in questo caso non ha impatto né su decentralizzazione, né sulla sicurezza. Infatti, ogni nodo contribuisce alla conferma delle transazioni con un ruolo scelto in modo efficiente mediante una selezione randomica, basata sulle \textit{Verifiable Random Function}. Inoltre, l'incremento del throughput delle transazioni non necessita di una diminuzione del numero di nodi aventi lo stesso ruolo, come avviene in approcci sharded. Il contributo di questa tesi dimostra da un punto di vista formale che è possibile risolvere il blockchain scalability trilemma.

La presente tesi è suddivisa in tre capitoli. Il primo capitolo presenta un'introduzione al tema delle blockchain, presentando la sua prima realizzazione, ovvero Bitcoin, per poi passare alle strutture dati autenticate e alle Distributed Hash Table, che saranno utili per la definizione dell'architettura proposta. Il secondo capitolo è dedicato allo stato dell'arte, presentando Algorand~\cite{gilad2017algorand}, una soluzione ai problemi di scalabilità degli algoritmi di consenso in approcci basati su Bitcoin, ponendo attenzione ad un approccio strutturale~\cite{bernardini2019blockchains} che consente ai nodi della rete di non dover memorizzare l'intero stato della blockchain, pur rimanendo in grado di validare le nuove transazione ed infine un analisi sulle architetture basate sullo sharding. Nel terzo capitolo sono analizzate le problematiche delle soluzioni correnti ed è presentata la nuova architettura proposta con questo lavoro di tesi, con i relativi teoremi di correttezza e scalabilità.